\documentclass[conference]{IEEEtran}

\input{preamble}

\newcommand{\C}{\EuScript{C}}
\newcommand{\E}{\EuScript{E}}
\newcommand{\Nee}{\EuScript{N}}

\newcommand{\btw}{\lessdot}
\newcommand{\mat}{\odot}
\newcommand{\ins}{\varpropto}
\newcommand{\g}{\mathfrak{g}}

\input{Theorem_eng}

% \usepackage{algorithmic}
% \usepackage{algorithm}
\usepackage{flushend}
\geometry{left = 3cm, right  =3cm, top = 4cm, bottom = 2cm, bindingoffset = 0cm}

\begin{document}

\title{The Structure of Convexity} 
\date{}%date stay empty

\author{
Roman Maksimovich (e-mail: r.a.maksimovich@gmail.com) \\
Supervisor: Arshak Aivazian, St. Petersburg State University student.\\ Academija Crnjanski, Serbia.
}
\maketitle


\section*{\Large{I. Introduction }}

The idea of a convex set is extremely renown and applied. We can observe it being utilized in functional analysis (Krein-Milman theorem), Eucledian geometry (Helly's theorem), convex analysis (convex functions), computer science (Jarvis algorithm), and many more. Furthermore, convexity can be generalized as a separate structure in the following way. A set \(X\) with a family \(\C\) of its subsets is called a \textit{convex space}, and elements of \(\C\) are called \textit{convex sets}, if

\begin{itemize}
    \item Both \(\varnothing\) and \(X\) lie in \(\C\);
    \item If \(\A \subset \C\) then \(\bigcap \A \in \C\);
    \item If \(\Nee \subset \C\) is a \textit{net} then \(\bigcup \Nee \in \C\).
\end{itemize}

Unfortunately, little academic attention is given to abstract convexity. The purpose of this work is to aquire an intuitive and formal understanding of, first, the internal concepts and connections in the theory of convex spaces, and, second, how this theory interrelates with other areas of mathematics.

The author of this research wishes to express exceptional gratitude to the Laboratory for Continuous Mathematical Education, St. Petersburg, where he was taught everything he knows.

\section*{\Large{II. Research methodology}}

Like most mathematical scientific works, our research is conducted by introducing definitions and then proving statements in order to unravel some properties of the defined concepts. There are three main branches of study in our work:

\textbf{Internal theory.} Here we explore the inner features of convexity theory. We introduce objects such as convex spaces, polytopes and hyperplanes and discover their interconnections.

\textbf{Inducing structures.} Here we examine systems that naturally produce convexity. Emerging properties of the resulting convex spaces help develop global convexity concepts like \textit{segmentiality} or \(n\)-\textit{affinity}.

\textbf{Induced structures.} Here we discover the role of convexity in specific contexts. In particular, we enter differential geometry with the general concept of a \textit{locally convex space}.
%the theory of Riemannian manifolds


\section*{\Large{III. Results}}

Our main results conclude to the following:

\begin{enumerate}
    \item[i] \textit{(internal theory)} The finite nature of convexity was discovered, as well as structural dimension properties of polytopes, sets and hyperplanes. The hyperfamily uniqueness theorem was proven. The Polytope Union Lemma was stated and shown to relate to hyperplanes in a fundamental way.
    
    \item[ii] \textit{(inducing structures)} The order convexity was defined and shown to always be free. A connection between linear and \(1\)-affine spaces was found, and hence the general \(n\)-affinity was derived.
    
    \item[iii] \textit{(inducing structures)} The metric convexity was defined and generalized to segmential convexity. Sufficient conditions for join-commutativity were found, as well as that for freedom.
    
    \item[iv] \textit{(induced structures)} The definition of local convexity was introduced. It was shown that \(B^2\) is not isomorphic to \(\R^2\), but is locally isomorphic. It was proven that riemannian manifolds are locally uniquely geodesic.
\end{enumerate}

Our subsidiary results are as follows:

\begin{enumerate}
    \item[i] The structural convex properties of uniquely geodesic metric spaces were researched.
    \item[ii] A procedure of inducing a topology on a specific kind of convex spaces was found. 
\end{enumerate}



\section*{\Large{ References }}

\begin{itemize}
    \item M.L.J. van de Vel, (1993) \textit{``Theory of convex structures''}, North-Holland mathematical library;
    \item David C. Kay, Eugine W. Womble, (1971) \textit{``Axiomatic convexity theory''}, Pacific Journal of Mathematics;
    \item D. Gromoll, W. Klingenberg and W. Meyer, (1968) \textit{``Riemannsche Geometrie im Grossen''}, Springer-Verlag, Berlin. 
\end{itemize}

\end{document}
